\section*{Research Aims}

This research aims to combat LLM hallucination and lack of specialised knowledge in domain-specific QA settings by integrating the knowledge graph embeddings through a fusion function. 
In addition, we aim to develop a methodology that surpasses conventional fine-tuning paradigms in terms of data dependency and parameter efficiency. 
To achieve our goals, we formulate three research questions to support our study: 
\vspace{-0.5cm}
\begin{itemize}
    \item[\textbf{RQ1}:] How to retrieve question-relevant information from a large-scale Knowledge Graph to augment an LLM?
    \item[\textbf{RQ2}:] How to inject the heterogeneous representations obtained from the Knowledge Graph to enrich the LLM representation space?
    \item[\textbf{RQ3}:] How to adapt a general-purpose LLM using a limited number of labelled data to answer domain-specific questions effectively?
\end{itemize}
\vspace{-0.3cm}

By addressing the above research questions, we aim to contribute existing literature as follows:\vspace{-0.3cm}
\begin{enumerate}
    \item Our research fills a crucial gap in the literature by investigating how to effectively leverage learnt representations from a KG to augment LLM-based question answering. Our research will thus contribute timely new knowledge on adapting general-purpose LLMs to specific contexts.

    \item We will propose a novel learning framework for KG-augmented LLM adaptation. This framework will be designed to be data-efficient for training and flexible for adapting open-source LLMs to a specific domain.

    \item Our proposed research will improve the applicability of LLMs by reducing the amount of domain-specific data required for effective training. It will provide insights into the adoption of LLMs across various businesses and application scenarios.
    
\end{enumerate}

