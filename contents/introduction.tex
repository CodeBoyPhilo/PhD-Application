\section*{Introduction}\label{sec:introduction}

Large Language Models (LLMs) are generative artificial intelligence (AI) systems designed to understand, generate, and interact with human language. 
LLMs have been attracting widespread attention from both experts and
the public, due to their advanced reasoning capabilities and exceptional performance across a variety of applications, such as text summarization, code generation, sentiment analysis, and question answering~\parencite{brown2020language, zero-shot-reasoner, palm, won2024scaling}.
With their abilities to reason and interact, more companies are incorporating LLMs into their business models to create added values for customers. 
For example, the fast food industry giant Domino's Pizza uses LLM-based chatbots within their recommender system
to assist customers in placing online orders~\parencite{dominos_ai}.
Generative AI technologies are expected to significantly impact various industry sectors, from banking, education, legal services, to healthcare. According to a report by McKinsey \& Company, the integration of LLM-based technologies is estimated to contribute over \$3.1 trillion additional value annually, more than the GDP of the United Kingdom in 2021~\parencite{mkinsey}.

Despite the high expectations for LLMs in various downstream applications, recent studies have identified two critical weaknesses that could undermine their reliability and trustworthiness in answering domain-specific questions.
The first critical issue is the frequent occurrence of \emph{hallucination} in LLM response generation
~\parencite{kgr, kalm-prompting, huang2023survey, ji2023survey}. 
Hallucinated responses often contain logical flaws and even factually inaccurate information.
This flaw significantly impairs the reasoning performance of LLMs and negatively impact the efficacy of LLMs in domain-specific question-answering (QA) tasks.
For example, consider a scenario where a user seeks biomedical advice from an LLM; receiving a response containing fabricated misinformation could potentially lead to life-threatening consequences~\parencite{pmc-llama}.
Another major weakness of LLMs in domain-specific question-answering is their \emph{lack of specialised domain knowledge}.
A general-purpose LLM is typically trained on massive datasets from a broad scope of publicly available information. They often lack the specialised expertise required for generating accurate and reliable responses in specialised domains.

To combat the challenges of \emph{hallucination} and \emph{lack of specialised knowledge}, a series of methods have been recently proposed, which can be classified into domain-specific fine-tuning methods and external resources augmented methods.

Domain-specific fine-tuning methods, as the name suggests, aim to \emph{edit} the internal knowledge of the LLM by updating the model parameter through a full-scale fine-tuning or parameter-efficient fine-tuning methods such as LoRA~\parencite{lora}.
However, these fine-tuning methods often require a large number of annotated data (e.g., one million tokens) in Question-Answer pairs or Instruction-Input-Output formats. In practice, such data is often scarce due to the high annotation cost. 
In addition, the fine-tuning methods also require high computational resources as the scale of LLM parameters increases.

On the other hand, external resources augmented methods explore the use of external resources such as databases, APIs, and Knowledge Graphs (KGs) to augment the LLM generation process. 
These methods retrieves top-$k$ relevant information from reliable sources and inject them into the LLM as contexts to enrich the response generation process.
However, these methods are susceptible to knowledge noise~\parencite{kbert}, i.e., not all retrieved knowledge is relevant to the target question.

Very recently, research studies have attempted to investigate the use of a domain-specific Knowledge Graph (KG) as the knowledge
provider to augment LLM response generation ~\parencite{gnp, graph-prompter, kalm-prompting}.
This line of research leverages knowledge triplets to represent factual information, represented as $(e_h, r, e_t)$ where $e_h, e_t$ are the head and tail entities, and $r$ represents the relation associated with them (e.g.\ [\texttt{Canberra, CapitalOf, Australia}]).
By utilising the representation power of LLMs and relational information contained in KGs, there is considerable potential to enhance knowledge representations via information exchange between KGs and LLMs.
While these methods are less data-dependent and often model-agnostic, finding out how to best fuse the heterogeneous information between KGs and LLMs for question-answering remains an open research challenge.

Based on the above observations, we intend to research the integration between KGs and LLMs for domain-specific question-answering tasks. 
At a higher level, we tackle LLM domain-specific question-answering tasks with the objectives of achieving data efficiency (low dependence on labelled training dataset) and parameter efficiency (small-scale of trainable model parameters).
To achieve this, first, we plan to utilise context-aware graph pruning methods to remove redundant knowledge from the KGs to the maximum extent to mitigate the drawbacks of knowledge noise. 
We will further investigate the current methods of the knowledge encoder modules to formulate a graph representation learning algorithm based on information theory. 
Finally, we research methods to effectively fuse the text modality and the graph modality together to enrich the representation space of the LLM using our learned knowledge embeddings.
